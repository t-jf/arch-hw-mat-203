\documentclass{article}
\usepackage[utf8]{inputenc}

\title{Calculus III Midterm 2 - Solutions}
\date{November 2019}

\begin{document}

\maketitle

\section{Surface Area of a Sphere}
\begin{enumerate}
	\item Cylindrically -- cylindrical bands on the surface
	    \newline \newline
        Use the gradient form
        \newline
            $\int\int
                \sqrt{
                    (\frac{\partial z}{\partial x})^2 +
                    (\frac{\partial z}{\partial y}^2) + 1
                } dA$$

        \newline
            $z = +-\sqrt{1-x^{2}-y^{2}}$
        \newline
            $\frac{\partial z}{\partial x} = \frac{-x}{\sqrt{1-x^2+y^2}}$
        \newline
            $\frac{\partial z}{\partial y} = \frac{-y}{\sqrt{1-x^2+y^2}}$
        \newline
            $0 <= \theta <= 2\pi $
        \newline
            $0 <= \phi <= \pi $
        \newline
            $x = \rho\sin(\phi)\cos(\theta) $
        \newline
            $y = \rho\sin(\phi)\sin(\theta) $
        \newline
            $z = \rho\cos(\phi) $
        \newline
            Jacobian$ = \rho^2\sin(\phi) $
        \newline
            $2\int_{0}^{2\pi}\int_{0}^{1} \frac{1}{\sqrt{1-r^2}}\cdot r dr d\theta$
        \newline
	\item Spherically -- rectangles on surface
        \newline \newline
            $2\int_{\theta=0}^{\theta=\pi}\int_{\phi=0}^{\phi=2\pi} (r sin \theta) d\phi d\theta$
        \newline \newline
	\item Stereographically -- vector intersections with the surface
\end{enumerate}

\section{Volume of a Sphere}
\begin{enumerate}
	\item Cylindrically -- discs stacking to form sphere
	\item Spherically -- convert from Cartesian and calculate
	\item Cylindrically -- convert from Cartesian and calculate
	\item Spherically -- order integral to form "shells" emanating from origin
	\item Stereographically -- vector intersections with the surface
\end{enumerate}

\section{Gaussian Normal Curve as a Probability Density Function}

\section{Sequences and Series}

\section{Using series to calculate a pesky integral}

\section{Using series to calculate $\pi$}

\section{Using series to calculate}

$\iint_{a}^{b} x^2 dx$

\end{document}
